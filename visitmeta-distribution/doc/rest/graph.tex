\section{Graph Management}
The last sections shows how to view graphs or deltas at different timestamps.
\subsection{Changes Map}
\begin{lstlisting}
HTTP:GET
http://example.com:8000/{Connection Name}/graph/changes
\end{lstlisting}

\begin{minipage}{\linewidth}
\textbf{Example Request:}
\begin{lstlisting}
HTTP:GET
http://example.com:8000/default/graph/changes
\end{lstlisting}
\end{minipage}

\begin{minipage}{\linewidth}
\textbf{Response:}
\begin{lstlisting}
{
    "1425915295000": 1,
    "1425915342000": 1
}
\end{lstlisting}
\end{minipage}
The Response is a JSON-Object mapping timestamps on the amount of changes occurred at that time.


\subsection{Initial Graph}\label{initial}
\begin{lstlisting}
HTTP:GET
http://example.com:8000/{Connection Name}/graph/initial
\end{lstlisting}

\begin{minipage}{\linewidth}
\textbf{Example Request:}
\begin{lstlisting}
HTTP:GET
http://example.com:8000/default/graph/initial
\end{lstlisting}
\end{minipage}

\begin{minipage}{\linewidth}
\textbf{Response:}
\begin{lstlisting}
[{
    "timestamp": 1425915295000,
    "links": [{
        "identifiers": [{
            "typename": "device",
            "properties": {
                "name": "freeradius-pdp"
            }
        }, {
            "typename": "access-request",
            "properties": {
                "name": "ar1"
            }
        }],
        "metadata": {
            "typename": "access-request-device",
            "properties": { 
                "ifmap-cardinality": "singleValue",
            }
        }
    }]
}]
\end{lstlisting}
Note: The response was reduced for an easier view.
\end{minipage}

\subsection{Current Graph}
\begin{lstlisting}
HTTP:GET
http://example.com:8000/{Connection Name}/graph/current
\end{lstlisting}

\begin{minipage}{\linewidth}
\textbf{Example Request:}
\begin{lstlisting}
HTTP:GET
http://example.com:8000/default/graph/current
\end{lstlisting}
\end{minipage}

\begin{minipage}{\linewidth}
\textbf{Response:}
See \ref{initial}
\end{minipage}

\subsection{Graph At}
\begin{lstlisting}
HTTP:GET
http://example.com:8000/{Connection Name}/
	graph/{Timestamp}
\end{lstlisting}

\begin{minipage}{\linewidth}
\textbf{Example Request:}
\begin{lstlisting}
HTTP:GET
http://example.com:8000/default/graph/1425915342000
\end{lstlisting}
\end{minipage}

\begin{minipage}{\linewidth}
\textbf{Response:}
See \ref{initial}
\end{minipage}

\subsection{Notifies At}
\begin{lstlisting}
HTTP:GET
http://example.com:8000/{Connection Name}/graph/
	{Timestamp}?onlyNotifies=true
\end{lstlisting}

\begin{minipage}{\linewidth}
\textbf{Example Request:}
\begin{lstlisting}
HTTP:GET
http://example.com:8000/default/
	graph/314159265?onlyNotifies=true
\end{lstlisting}
\end{minipage}

\textbf{Response:}
See \ref{initial}. Only difference to initial, current or graph at response: each notify metadata has its own subgraph.


\subsection{Delta}
\begin{lstlisting}
HTTP:GET
http://example.com:8000/{Connection Name}/graph/
	{Timestamp From}/{Timestamp To}
\end{lstlisting}

\begin{minipage}{\linewidth}
\textbf{Example Request:}
\begin{lstlisting}
HTTP:GET
http://example.com:8000/default/
	graph/314159265/358979323
\end{lstlisting}
\end{minipage}

\begin{minipage}{\linewidth}
\textbf{Response:}
See \ref{initial}
\end{minipage}

\subsection{Graph Filter}
Initial, Current and GraphAt responses may be filtered. Only necessary changes are HTTP:POST instead of HTTP:GET and a Content-Type: application/json containing the filter information.

\paragraph{startId}
Identifier where the filter begins the search (represented as JSON).
\paragraph{maxDepth}
Integer value determining the maximal amount of links traveled from the first Identifier.
\paragraph{resultFilter}
Filterstring that filters Metadata. If the resultFilter is empty, no Metadata will be filtered. If the resultFilter is null, all Metadata will be filtered, resulting in a set only containing Identifiers. The filterstring should follow the filter-syntax specified by ifmap.
\paragraph{matchLinks}
Filterstring that determines what Link-types are allowed in the filtered graph. If matchLinks is empty, all Link-types are allowed. If matchLinks is null, no Link-types are allowed resulting in a Graph just containing the start Identifier.

\begin{minipage}{\linewidth}
\textbf{Example Request:}
\begin{lstlisting}
HTTP:POST
Content-Type: application/json
http://example.com:8000/default/graph/initial
{
startId:
  {
    type: device,
    name: freeradius-pdp
  },
  maxDepth: 3,
  resultFilter: "meta:event/name=\"event1\"",
  matchLinks: "meta:device-ip"
}
\end{lstlisting}
\end{minipage}

\begin{minipage}{\linewidth}
\textbf{Response:}
See \ref{initial}
\end{minipage}
